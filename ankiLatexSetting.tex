\documentclass[12pt]{article}
\special{papersize=3in,5in}
\usepackage[utf8]{inputenc}
\usepackage{amssymb,amsmath,amsfonts,mathrsfs, lmodern, textcomp}
\usepackage{amsthm}
\usepackage{xcolor}
\usepackage{lipsum} % dummy text for the MWE
\usepackage{enumitem}
\usepackage[ampersand]{easylist}
\usepackage{tikz}
\usetikzlibrary{tikzmark,calc,decorations.pathreplacing,automata,positioning,arrows}

% Select what to do with todonotes: 
% \usepackage[disable]{todonotes} % notes not showed
\usepackage[draft]{todonotes}   % notes showed

% Select what to do with command \comment:  
% \newcommand{\comment}[1]{}  %comment not showed
\newcommand{\comment}[1]{\par {\bfseries \color{blue} #1 \par}} %comment showed

\pagestyle{empty}
\setlength{\parindent}{0in}
\newtheorem{theorem}{Theorem}[section]
\newtheorem{corollary}{Corollar}[theorem]
\newtheorem{lemma}[theorem]{Lemma}

\theoremstyle{definition}
\newtheorem{definition}{Definition}[section]

\theoremstyle{remark}
\newtheorem*{remark}{Remark}
 
\begin{document} % Start document
\section{Introduction}
Theorems can easily be defined
 
\begin{theorem}
Let $f$ be a function whose derivative exists in every point, then $f$ is 
a continuous function.
\end{theorem}
 
\begin{theorem}[Pythagorean theorem]
\label{pythagorean}
This is a theorema about right triangles and can be summarised in the next 
equation 
\[ x^2 + y^2 = z^2 \]
\end{theorem}
 
And a consequence of theorem \ref{pythagorean} is the statement in the next 
corollary.
 
\begin{corollary}
There's no right rectangle whose sides measure 3cm, 4cm, and 6cm.
\end{corollary}
 
You can reference theorems such as \ref{pythagorean} when a label is assigned.
 
\begin{lemma}
Given two line segments whose lengths are $a$ and $b$ respectively there is a 
real number $r$ such that $b=ra$.
\end{lemma}

Unnumbered theorem-like environments are also possible.
 
\begin{remark}
This statement is true, I guess.
\end{remark}
 
And the next is a somewhat informal definition
 
\theoremstyle{definition}
\begin{definition}{Fibration}
A fibration is a mapping between two topological spaces that has the homotopy lifting property for every space $X$.
\end{definition}

\begin{lemma}
Given two line segments whose lengths are $a$ and $b$ respectively there 
is a real number $r$ such that $b=ra$.
\end{lemma}
 
\begin{proof}
To prove it by contradiction try and assume that the statemenet is false,
proceed from there and at some point you will arrive to a contradiction.
\end{proof}

\end{document} % End document
